\documentclass[12pt]{article}
 \usepackage[margin=1in]{geometry} 
\usepackage{amsmath,amsthm,amssymb,amsfonts}
\usepackage{graphicx}
\usepackage{titlesec}
\titleformat*{\section}{\large\bfseries}
 
\newcommand{\N}{\mathbb{N}}
\newcommand{\Z}{\mathbb{Z}}
 
\newenvironment{problem}[2][]{\begin{trivlist}
\item[\hskip \labelsep {\bfseries #1}\hskip \labelsep {\bfseries #2.}]}{\end{trivlist}}

 
\begin{document}
  
\title{Computational Physics Project 1: Pendulum}
\author{Ben Zager, Remy Wang}
\maketitle

\section*{Part A}



\section*{Part B}
%%%%%% 1B %%%%%%
\begin{problem}{1}
	\textbf{Phase space of nonlinear pendulum}

	As $\theta_{0}$ approaches $\pi$, the trajectories go from an ellipse to a more "lemon" shape, as seen in figure \ref{phase}.

	For $\theta_{0} = 0$, varying $\dot{\theta}_{0} \in [0,\pi]$, the phase shows two different behaviors.  For $\dot{\theta}_{0}$ between 0 and approximately $0.6\pi$ the phase space appears similar to the previous one, as seen in figure \ref{phaseDot}a.  If $\dot{\theta}_{0}$ is greater than that, the motion is no longer periodic, and $\theta$ increases indefinitely, as seen in figure \ref{phaseDot}b.

\begin{figure}[ht!]
	\centering
	\includegraphics[scale=0.6]{../figures/phaseSpace.png}
	\caption{Plots of trajectory $(\theta,\dot{\theta})$, for many values of $\theta_{0} \in [0,\pi]$}
	3\label{phase}
\end{figure}

\begin{figure}[ht!]
	\centering
	\begin{minipage}[b]{0.4\textwidth}
		\includegraphics[scale=0.6]{../figures/phaseSpaceDot.png}
	\end{minipage}
	\hfill
	\begin{minipage}[b]{0.4\textwidth}
		\includegraphics[scale=0.6]{../figures/phaseSpaceDot2.png}
	\end{minipage}
	\caption{Trajectory for various $\dot{\theta}$}
	\label{phaseDot}
\end{figure}
\end{problem}

%%%%%% 2B %%%%%%
\begin{problem}{2}
	\textbf{Phase space of linear pendulum} \\
	For the linear pendulum, the phase space trajectory remains elliptical for all values of $\theta_{0}$, as seen in figure \ref{phaseLin}

\begin{figure}[h!]
\centering
  \includegraphics[scale=0.6]{../figures/phaseSpaceLinear.png}
  \caption{Plots of linearized trajectory $(\theta,\dot{\theta})$, for many values of $\theta_{0} \in [0,\pi]$}
  \label{phaseLin}
\end{figure}
\end{problem}

%%%%%% 3B %%%%%%
\begin{problem}{3}
	\textbf{Pendulum with driving force, $\gamma k^{2}cos(\omega t)$} \\
	If a periodic driving force with $\omega = k$ is added, the frequency stays the same, but the ampltude varies periodically, as seen in figure \ref{driving}. 
\begin{figure}[h!]
	\centering
  	\includegraphics[scale=0.6]{../figures/drivingForce.png}
 	\caption{Solution for driven undamped pendulum}
  	\label{driving}
\end{figure}
\end{problem}

%%%%%% 4B %%%%%%
\begin{problem}{4}
	\textbf{Exploration of driven system} \\
	For fixed $\theta$ and $\dot{\theta}$, how do the real and phase space trajectories vary with $\gamma$.  

	\begin{figure}[ht!]
	\centering
	\begin{minipage}[b]{0.4\textwidth}
		%\includegraphics[scale=0.6]{../figures/.png}
	\end{minipage}
	\hfill
	\begin{minipage}[b]{0.4\textwidth}
		%\includegraphics[scale=0.6]{../figures/.png}
	\end{minipage}
	\caption{}
	\label{diving2}
\end{figure}
\end{problem}

%%%%%% 5B %%%%%%
\begin{problem}{5}
	\textbf{Identifying $(\theta_{0},\gamma)$ for which the motion diverges} \\

	Figure \ref{diverge} shows the phase plot for $(\theta_{0},\gamma)$ for $\theta_{0} \in [0,\pi]$, and $\gamma \in [0,6]$, after a time interval of $8\pi$ seconds.  The white regions indicate values for which the motion remained periodic. The blue regions indicate values for which the motion diverged.  The darker the color, the greater the value of $\theta$ at the end of the time interval.
\begin{figure}[h!]
	\centering
  	\includegraphics[scale=0.5]{../figures/diverge2.png}
 	\caption{Phase plot for $(\theta_{0},\gamma)$. On the left is the actual data that was calculated, and on the right is an interpolated contour plot.}
  	\label{diverge}
\end{figure}
\end{problem}

%%%%%% 6B %%%%%%
\begin{problem}{6}
	\textbf{Driven pendulum with damping} $\ddot{\theta}+2\beta\dot{\theta}+k^{2}sin\theta=\gamma k^{2}cos(\omega t)$ \\

As $\gamma$ increases, the pendulum transitions to chaos, as seen in figure \ref{damped}.  This transition is known as \textit{period doubling}, which is becomes apparent from observing the plots.  

\begin{figure}[h!]
	\centering
  	\includegraphics[scale=0.5]{../figures/dampedDriven.png}
 	\caption{Plots of driven damped pendulum for various values of $\gamma \in [0,2]$}
  	\label{damped}
\end{figure}

\end{problem}

%%%%%% 7B %%%%%%
\begin{problem}{7}
	\textbf{Fourier analysis}
\end{problem}

\section*{Part C}

%%%%%% 1C %%%%%%
\begin{problem}{1}
\textbf{Visualization}

\end{problem}

%%%%%% 2C %%%%%%
\begin{problem}{2}
\textbf{Lyapunov exponent}
\end{problem}

%%%%%% 3C %%%%%%
\begin{problem}{3}
\textbf{Transition to Chaos}
\end{problem}

%%%%%% 4C %%%%%%
\begin{problem}{4}
\textbf{Time for Pendulum to Flip}
\end{problem}

\end{document}